\documentclass{QAstatement}

% build with pdflatex --shell-escape QAoverall.tex



\begin{document}
\title{QA Statement for Vibrothermography Model Development}
\author{Stephen D. Holland}
\date{December 4, 2012}

\maketitle

\section*{Motivation}
{\small (This section gives the big picture of why our process needs QA)} \\
The research program we have been asked to undertake is moderately 
high risk and on a compressed schedule. 
\begin{enumerate}
  \item There isn't time or funding to redo experiments an extra time in case we mess them up the first time. 
  \item Based on previous experiments we know the empirical portion of the 
model will have a moderately high variance. In case the variance is too 
high to be useful we need to have tightly controlled processes in order
to show that the variance is a result of the inherent physical processes
of vibrothermography as opposed to lousy experimental practices. 
  \item Since there is a chance that our results will be used in 
safety-critical lifing and POD estimation, it is our moral responsibility
to make sure that our results are accurate. 
\end{enumerate}
For these reasons we need quality assurance processes in place to make
sure that process steps are performed correctly and documented. 

\section*{Action}
{\small (This section lists specific QA needs and our 
approach to satisfying them. It can also be used to explain
how things work under the hood.)} \\
\begin{itemize}
\item We will document all repeatable processes or
  controlled experiments. We will keep records of completion of those process
  steps in case questions come up later. 
\item We will maximize the use of automation to eliminate the chance of 
human error, but we will still monitor the automatic processes to catch software bugs and equipment malfunctions.
% \item We will make note of anomalies. 
\item Lest files and knowledge become distributed across the team, we
  will store results in (or migrate results to) a central location
  that we will keep well organized.
\item So that our results are traceable, we will track changes and versions of processes, analysis scripts, etc. 
\item We will give each other feedback on our processes in both design and execution stages. 
\end{itemize}

\section*{Checklists}
{\small (This section lists specific actions one-by-one. Recommend 
writing it in XML so that it works with our automation tools.)} \\
(checklist begins on next page)
 
% \checklistxml: First parameter is identifier for this checklist
%                Second parameter is XML of this checklist
\checklistxml{chk}{<?xml version="1.0" encoding="UTF-8"?>
<?xml-stylesheet type="text/xsl" href="chx2html.xsl"?>
<!-- IMPORTANT... Edit this ONLY in QAoverall.tex, not the QAoverall.chx copy 
     which may be overwritten when QAoverall.tex is recompiled -->
<!-- Note: due to a bug in QAmanual.cls, blank lines are not allowed -->
<checklist xmlns="http://limatix.org/checklist" xmlns:chx="http://limatix.org/checklist"> 
    <!-- clinfo should be the base name of the .tex file, followed by
         an underscore and the checklist identifier, followed by an
         underscore and the version, followed by an underscore and the
         date. -->
    <clinfo>QAoverall_chk_V0.1_12/5/2012</clinfo>
    <cltitle>Checklist for writing quality assurance plans</cltitle>
    <checkitem>I have read the QA manual for Vibrothermography Model Development</checkitem>
    <checkitem>I have written a QA manual for my process that documents the rationale for the process, and specific QA needs. </checkitem>
    <checkitem>Attached to the QA manual for my process is a checklist.</checkitem>
    <checkitem>I have documented how and where the results and outputs from my process will be stored, and these files will make their way to the central 
repository. </checkitem>
    <checkitem>My process is automated to the extent practical</checkitem>
    <checkitem>My process includes steps to cross-check, to make sure the results are meaningful, and to make note of anomalies.</checkitem>
    <checkitem>Versions and changes of my process, scripts, etc. are
managed and tracked. If possible, script versions should be automatically
saved to data files.</checkitem>
    <checkitem>I have asked another team member for input on this process 
and its documentation, and I have improved it according to his or her input. </checkitem>
</checklist>
}

\end{document}
