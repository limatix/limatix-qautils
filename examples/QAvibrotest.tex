\documentclass{QAstatement}

% build with pdflatex --shell-escape QAvibrotest.tex



\begin{document}
\title{QA Statement for Vibrothermography Specimen Testing}
\author{Stephen D. Holland}
\date{December 5, 2012}

\maketitle

\section*{Motivation}
{\small (This section gives the big picture of how vibrothermography specimen
testing is done and why a QA process is necessary)} \\
Vibrothermography testing is a technique for finding cracks where a thermal 
camera is used to sense vibration-induce frictional heating of the crack surfaces. 

In planning a vibrothermography test we need to think about what sort of cracks 
we are looking for, where those cracks are likely to be, under what conditions 
we expect the surfaces of those cracks
to rub, how the specimen can be mounted, any need for applying an emissive coating, 
and the different orientations and 
camera views which will be required. 

Ideally the vibrothermography test would include some sort of coverage analysis 
step where we quantify the vibration field so as to predict crack detectability
as a function of position within the part. Right now we do not yet have a viable
coverage analysis procedure. 

The vibrothermography test starts with equipment setup and specimen mounting. The
specimen will be placed in a sequence of different orientations, and at 
each orientation will be tested over a sequence of excitation frequencies. 
The process is moderately tedious with manual and automatic steps interspersed. 
It is easy to get distracted and make a mistake. 

\section*{Action}
{\small (This section lists specific QA needs and our 
approach to satisfying them. )} \\
\begin{itemize}
\item Checklists for specimen setup, new orientation, and sequence of excitation frequencies
\item Use automation tools ({\tt dg\_checklist} and {\tt datacollect}) for simplifying repetitive tasks
\item Use version control system (Mercurial) on data collection and analysis scripts
\item Central location ({\tt /sata /sata2 /sata3}) for data storage.  
\item {\bf Need process for validating calibration of major components}
\end{itemize}

\section*{Checklists}
{\small (This section lists specific actions one-by-one. Recommend 
writing it in XML so that it works with our automation tools.)} \\
(checklist begins on next page)
 
% \checklistxml: First parameter is identifier for this checklist
%                Second parameter is XML of this checklist
\checklistxml{setup}{<?xml version="1.0" encoding="UTF-8"?>
<?xml-stylesheet type="text/xsl" href="chx2html.xsl"?>
<!-- IMPORTANT... Edit this ONLY in QAvibrotest.tex, not the QAvibrotest_setup.chx copy 
     which may be overwritten when QAvibrotest.tex is recompiled -->
<!-- Note: due to a bug in QAmanual.cls, blank lines are not allowed -->
<checklist  xmlns="http://limatix.org/checklist" xmlns:chx="http://limatix.org/checklist"> 
    <!-- clinfo should be the base name of the .tex file, followed by
         an underscore and the checklist identifier, followed by an
         underscore and the version, followed by an underscore and the
         date. -->
    <clinfo>QAvibrotest_setup_V1.0_12/5/2012</clinfo>
    <cltitle>Checklist for vibrothermography test setup</cltitle>
    <checkitem>Figure out sample mounting and coating issues
        <description>Verify that you have a good way to mount sample, transducer,
 and that the sample coating is adequate.</description>
    </checkitem>
    <checkitem>Figure out where thermal camera and vibrometers will be mounted.
        <description>Thermal camera usually wants to be as close as possible</description>
    </checkitem>
    <checkitem>Make sure dataguzzler is not already running
        <description>ps auxww |grep dataguzzler</description>
    </checkitem>
    <checkitem>Turn on waveform generator, laser vibrometer(s), and thermal camera
        <description>Turn on waveform generator, laser vibrometer(s), and thermal camera</description>
    </checkitem>
    <checkitem>run dataguzzler with appropriate config file
        <description>dataguzzler dataguzzler_vibrotherm.confm4</description>
    </checkitem>
    <checkitem>Check startup warnings for relevant errors
        <description>Look for errors relating to equipment that is turned on
or that you intend to use</description>
    </checkitem>
    <checkitem>Look for amber LED on image capture board. It should be solid, not flashing
        <description>If flashing, check camera trigger signal. If OK may need to power-cycle camera</description>
    </checkitem>
    <checkitem>Run dg_scope
        <description>dg_scope &amp;
Verify that camera frames are coming in (globalrev incrementing)</description>
    </checkitem>
    <checkitem>Load thermal camera calibration
        <description>e.g. dg_load_snapshot /home/shared/calib_sc6000_1009_2.3ms_102008_processed.dgs</description>
    </checkitem>
    <checkitem>Remove lens cap, verify reasonable temperature readout
        <description>Temperatures should be around room temperature (290-296 deg. K)</description>
    </checkitem>
    <checkitem>Release shutdown switches, verify bias generator red and green LEDs
        <description>Otherwise check shutdown switches and status of bias supply.</description>
    </checkitem>
    <checkitem class="adjustparam" title="Turn on pressure">
        <parameter type="str" name="description">Verify transducer coupling under pressure</parameter>
        <parameter type="str" name="dc-valuetype">numericunits</parameter>
        <parameter type="str" name="dc-valuedefunits">kPa</parameter>
        <parameter type="str" name="dg-param">PRESSURE</parameter>
        <parameter type="float" name="dc-valuethreshold">0.3</parameter>
        <parameter type="str" name="dg-paramdefault">100 kPa</parameter>
    </checkitem>
    <checkitem>Mount sample, transducer, and couplant
        <description>Verify that transducer contact is flat and over a wide area</description>
    </checkitem>
    <checkitem class="command" title="Turn on X, Y, and Z motion stages">
        <parameter type="str" name="dg-command">MOT:X:STATUS ON;MOT:Y:STATUS ON;MOT:Z:STATUS ON</parameter>
        <parameter type="str" name="description">Enable motion stages</parameter>
    </checkitem>
    <checkitem>Turn on power amplifier
        <description>right hand knob, all the way clockwise</description>
    </checkitem>
    <checkitem>Start datacollect
        <description>change directory to where you will store the the files
e.g. /sata/shared/whatever,
python /home/shared/datacollect/datacollect.py -c dc_&lt;configfile&gt;.py</description>
    </checkitem>
    <checkitem> Configure datacollect and fill in parameters
        <description>Name the test, enter a run number and variable, hit start.
Then fill in all relevant manual parameters.</description>
    </checkitem>
    <checkitem>Proceed with 'new orientation' checklist
        <description>./dg_checklist checklists/vibro_neworientation.chk &amp;</description>
    </checkitem>
</checklist>
}

\checklistxml{neworientation}{<?xml version="1.0" encoding="UTF-8"?>
<?xml-stylesheet type="text/xsl" href="chx2html.xsl"?>
<!-- IMPORTANT... Edit this ONLY in QAvibrotest.tex, not the QAvibrotest_setup.chx copy 
     which may be overwritten when QAvibrotest.tex is recompiled -->
<!-- Note: due to a bug in QAmanual.cls, blank lines are not allowed -->
<checklist  xmlns="http://limatix.org/checklist" xmlns:chx="http://limatix.org/checklist"> 
    <!-- clinfo should be the base name of the .tex file, followed by
         an underscore and the checklist identifier, followed by an
         underscore and the version, followed by an underscore and the
         date. -->
    <clinfo>QAvibrotest_neworientation_V1.0_12/5/2012</clinfo>
    <cltitle>Checklist for vibrothermography test at a new orientation</cltitle>
    <checkitem class="adjustparam" title="Move Z stage away">
        <parameter type="str" name="description">Move stage away (e.g. 80mm)</parameter>
        <parameter type="str" name="dc-valuetype">numericunits</parameter>
        <parameter type="str" name="dc-valuedefunits">mm</parameter>
        <parameter type="str" name="dg-param">MOT:Z</parameter>
        <parameter type="float" name="dc-valuethreshold">0.1</parameter>
        <parameter type="str" name="dg-paramdefault">80 mm</parameter>
    </checkitem>
    <checkitem class="text" title="Rotate stage to correct orientation">
        <parameter type="str" name="description">Rotate stage to desired orientation and secure it</parameter>
    </checkitem>
    <checkitem class="text" title="Correct view setting and set trial run">
        <parameter type="str" name="description">Change view setting in datacollect,
 set trial run mode</parameter>
    </checkitem>
    <checkitem class="multiparam" title="Move stages to new position">
        <parameter type="float" name="dc-valuethreshold2">0.1</parameter>
        <parameter type="str" name="description"></parameter>
        <parameter type="float" name="dc-valuethreshold1">0.1</parameter>
        <parameter type="str" name="dc-valuedefunits2">mm</parameter>
        <parameter type="str" name="dg-paramdefault2"> mm</parameter>
        <parameter type="str" name="dg-param1">MOT:X</parameter>
        <parameter type="str" name="dg-param2">MOT:Y</parameter>
        <parameter type="str" name="dc-valuedefunits1">mm</parameter>
        <parameter type="str" name="dc-valuetype1">numericunits</parameter>
        <parameter type="str" name="dc-valuetype2">numericunits</parameter>
        <parameter type="str" name="dg-paramdefault1"> mm</parameter>
    </checkitem>
    <checkitem class="adjustparam" title="Move stage closer">
        <parameter type="str" name="description">Move stage as close as possible (e.g. 125)</parameter>
        <parameter type="str" name="dc-valuetype">numericunits</parameter>
        <parameter type="str" name="dc-valuedefunits">mm</parameter>
        <parameter type="str" name="dg-param">MOT:Z</parameter>
        <parameter type="float" name="dc-valuethreshold">0.1</parameter>
        <parameter type="str" name="dg-paramdefault"> mm</parameter>
    </checkitem>
    <checkitem class="multiparam" title="Align vibrometer">
        <parameter type="float" name="dc-valuethreshold2">0.1</parameter>
        <parameter type="str" name="description">Align laser vibrometer and
apply vibrometer tape</parameter>
        <parameter type="float" name="dc-valuethreshold1">0.1</parameter>
        <parameter type="str" name="dc-valuedefunits2">mm</parameter>
        <parameter type="str" name="dg-paramdefault2"></parameter>
        <parameter type="str" name="dg-param1">MOT:X</parameter>
        <parameter type="str" name="dg-param2">MOT:Y</parameter>
        <parameter type="str" name="dc-valuedefunits1">mm</parameter>
        <parameter type="str" name="dc-valuetype1">numericunits</parameter>
        <parameter type="str" name="dc-valuetype2">numericunits</parameter>
        <parameter type="str" name="dg-paramdefault1"></parameter>
    </checkitem>
    <checkitem class="command" title="Focus laser vibrometer">
        <parameter type="str" name="dg-command">VIB:AUTOFOCUS;VIB:WAITAUTOFOCUS</parameter>
        <parameter type="str" name="description">Vibrometer autofocus</parameter>
    </checkitem>
    <checkitem class="adjustparam" title="Adjust vibrometer sensitivity">
        <parameter type="str" name="description">Adjust according to expected vibration amplitude</parameter>
        <parameter type="str" name="dc-valuetype">numericunits</parameter>
        <parameter type="str" name="dc-valuedefunits">mm</parameter>
        <parameter type="str" name="dg-param">VIB:RANGE</parameter>
        <parameter type="float" name="dc-valuethreshold">0.1</parameter>
        <parameter type="str" name="dg-paramdefault">1000 mm/s/V</parameter>
    </checkitem>
    <checkitem class="adjustparam" title="Focus camera">
        <parameter type="str" name="description">Focus camera. Set PIXELSPERINCH if applicable.</parameter>
        <parameter type="str" name="dc-valuetype">numericunits</parameter>
        <parameter type="str" name="dc-valuedefunits">Pixels/Inch</parameter>
        <parameter type="str" name="dg-param">PIXELSPERINCH</parameter>
        <parameter type="float" name="dc-valuethreshold">0.1</parameter>
        <parameter type="str" name="dg-paramdefault">INVALID</parameter>
    </checkitem>
    <checkitem class="multiparam" title="Pressure on and output on">
        <parameter type="str" name="description">Pressure and output</parameter>
        <parameter type="float" name="dc-valuethreshold1">0.3</parameter>
        <parameter type="str" name="dc-valuedefunits2"></parameter>
        <parameter type="str" name="dg-paramdefault2">ON</parameter>
        <parameter type="str" name="dg-param1">PRESSURE</parameter>
        <parameter type="str" name="dg-param2">AWG:OUTPUT</parameter>
        <parameter type="str" name="dc-valuedefunits1">kPa</parameter>
        <parameter type="str" name="dc-valuetype1">numericunits</parameter>
        <parameter type="str" name="dc-valuetype2">string</parameter>
        <parameter type="str" name="dg-paramdefault1">100 kPa</parameter>
    </checkitem>
    <checkitem class="command" title="1 kHz - 60 kHz sweep">
        <parameter type="str" name="dg-command">GEN:SWEEP Arb 1 kHz 60 kHz .2 s .21 s 1.19 s 1.2 s</parameter>
        <parameter type="str" name="description">Switch to broadband sweep</parameter>
    </checkitem>
    <checkitem class="adjustparam" title="Set amplitude">
        <parameter type="str" name="description">Set excitation amplitude
Wear earmuffs.</parameter>
        <parameter type="str" name="dc-valuetype">numericunits</parameter>
        <parameter type="str" name="dc-valuedefunits">V</parameter>
        <parameter type="str" name="dg-param">AWG:AMPL</parameter>
        <parameter type="float" name="dc-valuethreshold">0.01</parameter>
        <parameter type="str" name="dg-paramdefault">3.0</parameter>
    </checkitem>
    <checkitem class="command" title="Trigger">
        <parameter type="str" name="dg-command">TRIGGER</parameter>
        <parameter type="str" name="description">Issue system trigger </parameter>
    </checkitem>
    <checkitem class="command" title="Save vibrometer data from sweep">
        <parameter type="str" name="dg-command">WFM:COPY Vibrometer VibSweep;WFM:COPY Vibfft VibSweepFFT</parameter>
        <parameter type="str" name="description">Save in VibSweep/VibSweepFFT channels</parameter>
    </checkitem>
    <checkitem class="text" title="Determine resonances">
        <parameter type="str" name="description">Save resonant frequencies in notes and on scratch pad.</parameter>
    </checkitem>
    <checkitem class="text" title="Archive data">
        <parameter type="str" name="description">Save dgs and settings files using datacollect</parameter>
    </checkitem>
    <checkitem class="text" title="Test each resonance">
        <parameter type="str" name="description">Run test checklist at each observed resonance.</parameter>
    </checkitem>
</checklist>
}

\checklistxml{testfreqs}{<?xml version="1.0" encoding="UTF-8"?>
<?xml-stylesheet type="text/xsl" href="chx2html.xsl"?>
<!-- IMPORTANT... Edit this ONLY in QAvibrotest.tex, not the QAvibrotest_setup.chx copy 
     which may be overwritten when QAvibrotest.tex is recompiled -->
<!-- Note: due to a bug in QAmanual.cls, blank lines are not allowed -->
<checklist  xmlns="http://limatix.org/checklist" xmlns:chx="http://limatix.org/checklist"> 
    <!-- clinfo should be the base name of the .tex file, followed by
         an underscore and the checklist identifier, followed by an
         underscore and the version, followed by an underscore and the
         date. -->
    <clinfo>QAvibrotest_testfreqs_V1.0_12/5/2012</clinfo>
    <cltitle>Checklist for vibrothermography multi-frequency test</cltitle>
    <checkitem class="text" title="Switch measurement type to sequence">
        <parameter type="str" name="description">Set measurement type to sequence in datacollect</parameter>
    </checkitem>
    <checkitem class="sweepfreq" title="Set sweep frequency band">
        <parameter type="str" name="dg-timestring">0.2 s 0.21 s 1.19 s 1.2 s</parameter>
        <parameter type="str" name="freqdefault2">21000 Hz</parameter>
        <parameter type="str" name="wfmname">Arb</parameter>
        <parameter type="float" name="dc-valuethreshold">0.01</parameter>
        <parameter type="str" name="dc-valuedefunits">Hz</parameter>
        <parameter type="str" name="genmodule">GEN</parameter>
        <parameter type="str" name="freqdefault1">20000 Hz</parameter>
        <parameter type="str" name="description"></parameter>
    </checkitem>
    <checkitem class="command" title="Trigger sweep">
        <parameter type="str" name="dg-command">TRIGGER</parameter>
        <parameter type="str" name="description">Issue system trigger </parameter>
    </checkitem>
    <checkitem class="burstfreq" title="Identify resonance peak and set burst frequency">
        <parameter type="str" name="dg-timestring">0.2 s 0.21 s 1.19 s 1.2 s</parameter>
        <parameter type="str" name="wfmname">Arb</parameter>
        <parameter type="str" name="dc-valuedefunits">Hz</parameter>
        <parameter type="str" name="freqdefault">10000 Hz</parameter>
        <parameter type="str" name="genmodule">GEN</parameter>
        <parameter type="float" name="dc-valuethreshold">0.01</parameter>
        <parameter type="str" name="description"></parameter>
    </checkitem>
    <checkitem class="command" title="Trigger burst">
        <parameter type="str" name="dg-command">TRIGGER</parameter>
        <parameter type="str" name="description">Issue system trigger </parameter>
    </checkitem>
    <checkitem class="text" title="Check vibrometer amplitude">
        <parameter type="str" name="description">Check that vibration amplitude is adequate</parameter>
    </checkitem>
    <checkitem class="text" title="Check for indications">
        <parameter type="str" name="description">Carefully examine VibroFitImg and DiffStack.
Note down anything observed.</parameter>
    </checkitem>
    <checkitem class="text" title="Save data">
        <parameter type="str" name="description">Save dgs and settings file with datacollect</parameter>
    </checkitem>
</checklist>
}

\end{document}
